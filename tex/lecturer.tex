\documentclass[11pt,a4paper,sans]{moderncv} % Font sizes: 10, 11, or 12; paper sizes: a4paper, letterpaper, a5paper, legalpaper, executivepaper or landscape; font families: sans or roman

\moderncvstyle{casual} % CV theme - options include: 'casual' (default), 'classic', 'oldstyle' and 'banking'
\moderncvcolor{green} % CV color - options include: 'blue' (default), 'orange', 'green', 'red', 'purple', 'grey' and 'black'

\usepackage{lipsum} % Used for inserting dummy 'Lorem ipsum' text into the template
\usepackage{color}

\usepackage[scale=0.92]{geometry} % Reduce document margins
%\setlength{\hintscolumnwidth}{3cm} % Uncomment to change the width of the dates column
%\setlength{\makecvtitlenamewidth}{10cm} % For the 'classic' style, uncomment to adjust the width of the space allocated to your name

\usepackage[resetlabels,labeled]{multibib}
\newcites{Journals}{Journals}
\newcites{Chapters}{Book chapters}
\newcites{Conferences}{Conference proceedings}

%----------------------------------------------------------------------------------------
%	NAME AND CONTACT INFORMATION SECTION
%----------------------------------------------------------------------------------------

\firstname{\Huge Leonardo C. T.} % Your first name
\familyname{\Huge Bezerra} % Your last name
% All information in this block is optional, comment out any lines you don't need
% \title{Research curriculum}
\address{Digital Metropolis Institute (IMD), Federal University of Rio Grande do Norte (UFRN)}{leonardo.bezerra@ufrn.br}
% \mobile{+55 (84) 9 8892 0288}
% \email{leonardo.bezerra@ufrn.br}
% \photo[200px][0.4pt]{us.jpg} 
% The first bracket is the picture height, the second is the thickness of the frame around the picture (0pt for no frame)
%\quote{"Quite often in nature, when waves are reflected from obstacles, not all wave energy is reflected: some is \textcolor{olive}{absorbed} by the obstacle and some is \textcolor{olive}{transmitted past} the obstacle."
%\newline
%- \textcolor{olive}{Dean and Dalrymple}}

%----------------------------------------------------------------------------------------

\begin{document}

\makecvtitle % Print the CV title

\vspace{-1.cm}

%----------------------------------------------------------------------------------------
%	INTERESTS
%----------------------------------------------------------------------------------------


% committed

% adaptable

% enthusiastic

% proficient in major programming languages 

% in particular Python and R 

% software development principles

% data science pipelines

% cloud-based technologies 

% adapt to teaching from level 3 to level 7

% data collection, cleaning, and processing 

% building and deploying models

% machine learning/AI or/and advanced analytics (simulation, optimisation, forecasting, etc.)

% applications to real world problems in business and industry, and the public sector.

% considerable experience and demonstrated innovation in

% - HE teaching

% - academic leadership

% - academic team management 

% - curriculum development

% enabling and facilitating 

% - external industrial relationships

% - research grant applications

% - knowledge exchange


% face-to-face and online

% setting assessment, marking, and giving feedback

% supervision of final year and post-graduate projects

% acting as a personal tutor to provide pastoral support to students

% support students for competitions

% work based learning 

% other opportunities for enterprise

% School strategic subject based

% professional or pedagogic research and development

% other scholarly activities in order to support your teaching

% develop their external research profile 

% contribute to one or more of the School’s multidisciplinary research groups

\section{Teaching}

My current position at Federal University of Rio Grande do Norte~(UFRN) as a Lecturer in Big Data has provided me with the opportunity to help (re)formulate different educational programs (B.Sc. and M.Sc.) and supervise talented B.Sc., postgraduate apprenticeship, and M.Sc. students. I have also successfully co-supervised a Ph.D. student, since I was not allowed to be a main Ph.D. supervisor yet. Importantly, I have helped propose A.I./data science tracks for both the B.Sc. and M.Sc. programs, as well as a novel professional doctorate program in information technology. Finally, I have taken a proactive approach to learning, having designed and delivered 19 different course modules, which can be broadly categorized into computational thinking (ranging from abstract data types to programming interview preparation) and A.I./data science (with a focus on non-programmers). All of these course modules were designed with an emphasis on self-taught learning, which made them ideal for the COVID-19 pandemic period (and online learning thereof). In addition to having published a book chapter about the methodology I have employed, I have also partnered with Huawei Telecommunications in Brazil to provide an HCIA-AI deep learning cerfiticate training program which was offered online.% to over 200 students.

% \cvlistdoubleitem{computational intelligence}{data science}
% \cvlistdoubleitem{artificial intelligence}{accountability}

% Doutor em Engenharia e Tecnologia pela Université Libre de Bruxelles (ULB, Bélgica), possui experiência em pesquisa relacionada à Inteligência Computacional e sua aplicação em Ciência de Dados. Desde 2017, atua na Universidade Federal do Rio Grande do Norte (UFRN) como professor adjunto na área de Big Data. Entre 2018 e 2022, foi membro permanente do Programa de Pós-Graduação em Tecnologia da Informação (PPgTI, conceito CAPES 4), onde orientou diferentes dissertações de mestrado profissional em colaboração com grupos de pesquisa internacionais e empresas. Desde 2022, é membro permanente do Programa de Pós-Graduação em Sistemas e Computação (PPgSC, conceito CAPES 5). Além de publicações nos principais veículos científicos nos campos da visão computacional, NLP e Aprendizado de Máquina automatizado, tem atuado em projetos junto ao poder público e à iniciativa privada. No campo da mineração de dados, ajudou a coordenar a revisão e atualização da Cine Brasil junto ao INEP, propondo a versão 2018, formulando sua tabela de equivalência unificada com a versão 2000 e a aplicando a todos os cursos de graduação e sequenciais do Brasil. No campo da análise de agrupamentos, ajudou a segmentar os clientes da rede de supermercados Nordestão, a terceira maior rede do Nordeste. No campo da análise de séries temporais, ajudou a modelar as vendas varejistas dessa rede de supermercados, bem como a ocorrência de crimes junto à Secretaria de Segurança Pública do Rio Grande do Norte. Atualmente, atua junto ao TRF5 e ao grupo Neoenergia para formação de recursos humanos e orientação de projetos nos âmbitos do Poder Judiciário e da indústria de energia. Ambos os projetos são oriundos de atuação prévia, respectivamente a atuação docente na Residência em Tecnologia da Informação e a organização de um hackathon propondo painéis de monitoramento da produção de energia eólica e otimização da recomposição das redes de distribuição de energia.



%----------------------------------------------------------------------------------------
%	HISTORY SECTION
%----------------------------------------------------------------------------------------
\section{Educational programs}

\subsection{Doctorate}
\cventry{Proponent}{Information technology}{Professional}{Federal University of Rio Grande do Norte}{Natal, RN, Brazil}{}
\cventry{2022-}{Systems and computing}{DSc}{Federal University of Rio Grande do Norte}{Natal, RN, Brazil}{}

\subsection{Masters}
\cventry{2022-}{Systems and computing}{MSc}{Federal University of Rio Grande do Norte}{Natal, RN, Brazil}{}
\cventry{2018-2021}{Information technology}{Professional}{Federal University of Rio Grande do Norte}{Natal, RN, Brazil}{}

\subsection{Postgraduate apprenticeship}
\cventry{2021-2024}{Information technology}{5th Region Federal Regional Court (TRF5)}{Recife, PE, Brazil}{}{}
\cventry{2018-2024}{Information technology}{Federal Justice Section of Rio Grande do Norte (JFRN)}{Natal, RN, Brazil}{}{}
\cventry{2018-2024}{Information technology}{Regional Electoral Court of Rio Grande do Norte (TRE-RN)}{Natal, RN, Brazil}{}{}
\cventry{2019-2021}{Information technology}{Court of Accounts of the State of Rio Grande do Norte (TCE-RN)}{Natal, RN, Brazil}{}{}

\subsection{Undergratuate}
\cventry{2017-}{Information technology}{BSc}{Federal University of Rio Grande do Norte}{Natal, RN, Brazil}{}{}
\cventry{2017-}{Computer science}{BSc}{Federal University of Rio Grande do Norte}{Natal, RN, Brazil}{}{}
\cventry{2017-}{Computer engineering}{BEng}{Federal University of Rio Grande do Norte}{Natal, RN, Brazil}{}{}
\cventry{2016-2017}{Mathematical computing}{BSc}{Federal University of Paraíba}{João Pessoa, PB, Brazil}{}{}
\cventry{2016-2017}{Computer science}{BSc}{Federal University of Paraíba}{João Pessoa, PB, Brazil}{}{}
\cventry{2016-2017}{Computer engineering}{BEng}{Federal University of Paraíba}{João Pessoa, PB, Brazil}{}{}

\section{Supervision (apprenticeship and undergraduate)}

\subsection{Postgraduate apprenticeship}

\cventry{2021}{Autonomous income in the municipalities of the Rio Grande do Norte state}{Postgraduate apprenticeship on Information technology}{Paulo Roberto Oliveira de Melo}{Court of Accounts of the State of Rio Grande do Norte (TCE-RN), Federal University of Rio Grande do Norte (UFRN), Brazil}{}

\cventry{2020}{A data-driven approach to aid payment auditing in the Court of Accounts of the State of Rio Grande do Norte}{Postgraduate apprenticeship on Information technology}{Gabriel Felipe Azevedo de Sousa}{Court of Accounts of the State of Rio Grande do Norte (TCE-RN), Federal University of Rio Grande do Norte (UFRN), Brazil}{}

\cventry{2020}{Business intelligence to aid electoral accounts auditing in Rio Grande do Norte}{Postgraduate apprenticeship on Information technology}{Giuliard Cosmo Rodrigues}{Regional Electoral Court of Rio Grande do Norte (TRE-RN), Federal University of Rio Grande do Norte (UFRN), Brazil}{}

\cventry{2020}{Workflow management tool review for ETL orchestration at TRE-RN}{Postgraduate apprenticeship on Information technology}{Thiago de Oliveira}{Regional Electoral Court of Rio Grande do Norte (TRE-RN), Federal University of Rio Grande do Norte (UFRN), Brazil}{}

\subsection{Undergraduate (final year)}

\cventry{2021}{Comparing contextual embeddings for the semantic textual similarity in Portuguese}{BEng in Computer engineering}{José Estevam de Andrade Junior}{Federal University of Rio Grande do Norte (UFRN), Brazil}{}

\cventry{2021}{Assessing irace for automated machine learning}{BSc in Computer science}{Carlos Eduardo Morais Vieira}{Federal University of Rio Grande do Norte (UFRN), Brazil}{}

\cventry{2019}{Using artificial intelligence to aid depression detection}{BEng in Computer engineering}{Deângela Caroline Gomes Neves}{Federal University of Rio Grande do Norte (UFRN), Brazil}{}

\section{Course modules}

\subsection{Computational thinking}

\cventry{2023-}{Abstract data types}{BSc in Information technology}{Federal University of Rio Grande do Norte (UFRN)}{Natal, RN, Brazil}%
{I helped design and deliver this module on the definition and applications of abstract data types. In detail, the original course curriculum only comprised algorithms and data structures, and had an above 50\% failure rate. Introducing abstract data types prior to the data structures used to implement them significantly improved the success rate of the course, often surpassing 75\%. This is in part due to student evaluation, which is in part performed via coding interviews.}

\cventry{2023-}{Fundamentals of multi-paradigm, high-level software development}{BSc in Information technology}{Federal University of Rio Grande do Norte (UFRN)}{Natal, RN, Brazil}%
{I helped design this module on the fundamentals of software development using multiple programming paradigms and high-level programming languages. In detail, the original course curriculum (i)~only comprised computer programming, disregarding critical software development fundamentals such as testing and versioning; (ii)~only adopted the procedural and object-oriented programming paradigms, both from purist perspectives, and; (iii)~adopted low-level programming languages. Altogether, these factors hindered student engagement as courses were seen as unrealistic, leading to an above 50\% failure rate. Introducing software development fundamentals, how to benefit from multiple programming paradigms, and adopting high-level languages comprise a significantly improvement that is expected to raise the success rate of students.}

\cventry{2019-2019}{Coding interview preparation}{BSc in Information technology}{Federal University of Rio Grande do Norte (UFRN)}{Natal, RN, Brazil}%
{I helped design and deliver this introductory module on coding interviewing, which was created to help students prepare for real-world job interviews. The course comprised invited talks from alumni who had extensive experience in the industry (and interviewing) and coding interview problems and practice, which included invited recruiters from companies. Though experimental, this module was very important and timely as shortly after its delivery the demand for information technology professionals surged as a function of the pandemic.}

\cventry{2018-}{Computational thinking}{BSc in Information technology}{Federal University of Rio Grande do Norte (UFRN)}{Natal, RN, Brazil}%
{I helped design and deliver this introductory module on computational thinking. In detail, the original course curriculum introduced computer programming using a low-level textual programming language, hindering the problem-solving learning ability of the students and leading to an above 50\% failure rate. Conversely, this module introduces computational thinking from a gamified visual programming perspective, which later evolves to high-level textual programming. Using these approaches, the course is able to focus on problem-solving through computational thinking, and the success rate of this module is often above 75\%.}

\cventry{2017-}{Algorithms and data structures}{BSc in Information technology}{Federal University of Rio Grande do Norte (UFRN)}{Natal, RN, Brazil}%
{I helped design and deliver this course .}

\cventry{2017-2018}{Programming languages}{BSc in Information technology}{Federal University of Rio Grande do Norte (UFRN)}{Natal, RN, Brazil}{Something about this}

\cventry{2016-2017}{Design and analysis of algorithms}{BSc in Computer science}{Federal University of Paraíba (UFPB)}{Paraíba, PB, Brazil}{Something about this}

\cventry{2016-2017}{The scientific method}{BSc in Mathematical computing}{Federal University of Paraíba (UFPB)}{Paraíba, PB, Brazil}{Something about this}

\cventry{2016-2017}{Graph applications}{BSc in Mathematical computing}{Federal University of Paraíba (UFPB)}{Paraíba, PB, Brazil}{Something about this}

\end{document}