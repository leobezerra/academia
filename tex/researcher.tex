%%%%%%%%%%%%%%%%%%%%%%%%%%%%%%%%%%%%%%%%%
% "ModernCV" CV and Cover Letter
% LaTeX Template
% Version 1.11 (19/6/14)
%
% This template has been downloaded from:
% http://www.LaTeXTemplates.com
%
% Original author:
% Xavier Danaux (xdanaux@gmail.com)
%
% License:
% CC BY-NC-SA 3.0 (http://creativecommons.org/licenses/by-nc-sa/3.0/)
%
% Important note:
% This template requires the moderncv.cls and .sty files to be in the same 
% directory as this .tex file. These files provide the resume style and themes 
% used for structuring the document.
%
%%%%%%%%%%%%%%%%%%%%%%%%%%%%%%%%%%%%%%%%%

%----------------------------------------------------------------------------------------
%	PACKAGES AND OTHER DOCUMENT CONFIGURATIONS
%----------------------------------------------------------------------------------------

\documentclass[11pt,a4paper,sans]{moderncv} % Font sizes: 10, 11, or 12; paper sizes: a4paper, letterpaper, a5paper, legalpaper, executivepaper or landscape; font families: sans or roman

\moderncvstyle{casual} % CV theme - options include: 'casual' (default), 'classic', 'oldstyle' and 'banking'
\moderncvcolor{green} % CV color - options include: 'blue' (default), 'orange', 'green', 'red', 'purple', 'grey' and 'black'

\usepackage{lipsum} % Used for inserting dummy 'Lorem ipsum' text into the template
\usepackage{color}

\usepackage[scale=0.92]{geometry} % Reduce document margins
%\setlength{\hintscolumnwidth}{3cm} % Uncomment to change the width of the dates column
%\setlength{\makecvtitlenamewidth}{10cm} % For the 'classic' style, uncomment to adjust the width of the space allocated to your name

\usepackage[resetlabels,labeled]{multibib}
\newcites{Journals}{Journals}
\newcites{Chapters}{Book chapters}
\newcites{Conferences}{Conference proceedings}

%----------------------------------------------------------------------------------------
%	NAME AND CONTACT INFORMATION SECTION
%----------------------------------------------------------------------------------------

\firstname{\Huge Leonardo C. T.} % Your first name
\familyname{\Huge Bezerra} % Your last name
% All information in this block is optional, comment out any lines you don't need
% \title{Research curriculum}
\address{Digital Metropolis Institute (IMD), Federal University of Rio Grande do Norte (UFRN)}{leonardo.bezerra@ufrn.br}
% \mobile{+55 (84) 9 8892 0288}
% \email{leonardo.bezerra@ufrn.br}
% \photo[200px][0.4pt]{us.jpg} 
% The first bracket is the picture height, the second is the thickness of the frame around the picture (0pt for no frame)
%\quote{"Quite often in nature, when waves are reflected from obstacles, not all wave energy is reflected: some is \textcolor{olive}{absorbed} by the obstacle and some is \textcolor{olive}{transmitted past} the obstacle."
%\newline
%- \textcolor{olive}{Dean and Dalrymple}}

%----------------------------------------------------------------------------------------

\begin{document}

\makecvtitle % Print the CV title

\vspace{-1.cm}

%----------------------------------------------------------------------------------------
%	INTERESTS
%----------------------------------------------------------------------------------------

% \section{Interest}

My thesis on \textbf{computational intelligence (CI)} was seminal to my current research on \textbf{data science (DS)}, \textbf{artificial intelligence~(AI)}, and their impact on \textbf{socially relevant problems}. Concerning DS, I supervise applied research projects with both the public and private sectors. Partners include the Brazilian Judicial Branch and Ministry of Education, as well as regional, national, and multi-national companies in fields as diverse as retail, telecommunications, and energy. Regarding AI, I supervise graduate students on theses involving deep and automated machine learning, as well as the intersection of multi-objective optimization with other CI domains, such as multi-dimensional visualization and dynamic optimization. Finally, regarding socially revelant problems, I have assisted in the fight against the COVID-19 pandemic through science publication and communication, in attempt to counter the intensive disinformation campaign held in Brazil. 

% \cvlistdoubleitem{computational intelligence}{data science}
% \cvlistdoubleitem{artificial intelligence}{accountability}

% Doutor em Engenharia e Tecnologia pela Université Libre de Bruxelles (ULB, Bélgica), possui experiência em pesquisa relacionada à Inteligência Computacional e sua aplicação em Ciência de Dados. Desde 2017, atua na Universidade Federal do Rio Grande do Norte (UFRN) como professor adjunto na área de Big Data. Entre 2018 e 2022, foi membro permanente do Programa de Pós-Graduação em Tecnologia da Informação (PPgTI, conceito CAPES 4), onde orientou diferentes dissertações de mestrado profissional em colaboração com grupos de pesquisa internacionais e empresas. Desde 2022, é membro permanente do Programa de Pós-Graduação em Sistemas e Computação (PPgSC, conceito CAPES 5). Além de publicações nos principais veículos científicos nos campos da visão computacional, NLP e Aprendizado de Máquina automatizado, tem atuado em projetos junto ao poder público e à iniciativa privada. No campo da mineração de dados, ajudou a coordenar a revisão e atualização da Cine Brasil junto ao INEP, propondo a versão 2018, formulando sua tabela de equivalência unificada com a versão 2000 e a aplicando a todos os cursos de graduação e sequenciais do Brasil. No campo da análise de agrupamentos, ajudou a segmentar os clientes da rede de supermercados Nordestão, a terceira maior rede do Nordeste. No campo da análise de séries temporais, ajudou a modelar as vendas varejistas dessa rede de supermercados, bem como a ocorrência de crimes junto à Secretaria de Segurança Pública do Rio Grande do Norte. Atualmente, atua junto ao TRF5 e ao grupo Neoenergia para formação de recursos humanos e orientação de projetos nos âmbitos do Poder Judiciário e da indústria de energia. Ambos os projetos são oriundos de atuação prévia, respectivamente a atuação docente na Residência em Tecnologia da Informação e a organização de um hackathon propondo painéis de monitoramento da produção de energia eólica e otimização da recomposição das redes de distribuição de energia.



%----------------------------------------------------------------------------------------
%	HISTORY SECTION
%----------------------------------------------------------------------------------------
\section{History}

\subsection{Appointments}
\cventry{2017-}{Assistant professor}{Federal University of Rio Grande do Norte}{Natal, RN, Brazil}{}{}
\cventry{2016-2017}{Assistant professor}{Federal University of Paraíba}{João Pessoa, PB, Brazil}{}{}

\subsection{Awards}
\cventry{2011-2016}{Ph.D. degree in Engineering and Technology}{Université Libre de Bruxelles}{Brussels, Belgium}{}{A component-wise approach to multi-objective evolutionary algorithms: from flexible frameworks to automatic design}
% \cventry{2009-2011}{M.Sc. in Systems and Computing}{Federal University of Rio Grande do Norte}{Natal, RN, Brazil}{}{}
% \cventry{2004-2008}{B.Sc. in Computer Science}{Federal University of Rio Grande do Norte}{Natal, RN, Brazil}{}{}

% \subsection{Junior Spring Semester Project}

% \cventry{2018}{Flow Mapping (Particle Image Velocimetry)}{Research Experiment}{}{}{This was an experiment in fluid dynamics with respect to a cylinder subjected to a constant flow. The water is seeded with small particles that are illuminated by a powerful laser. A cylinder is moved mechanically in various patterns mimicking theoretical 'free vibration' due to vortex shedding. The images are processed and analyzed in MATLAB and the resulting turblence/vorticity can be visualized. The forces experienced by the body are widely applicable (energy generation, structural pile requirements). Results were used in graduate student Erdem Aktosun's research.}

%----------------------------------------------------------------------------------------
%	AWARDS SECTION
%----------------------------------------------------------------------------------------

% \subsection{Achievements}
\cventry{2011}{F.R.I.A doctoral fellowship}{Fonds de la Recherche Scientifique (FNRS)}{Brussels, Belgium}{}{}
\cventry{2009}{Best paper award}{Brazilian Symposium on Augmented and Virtual Reality (SVR)}{Porto Alegre, Brazil}{}{}

%----------------------------------------------------------------------------------------
%	PROJECTS
%----------------------------------------------------------------------------------------

\section{Projects and funding}

\cventry{2023-2024}{Information technology postgraduate apprenticeship (class of 2024)}{5th Region Federal Regional Court (TRF5), Brazil}{R\$3,500,891.11}{Collaborator}{}

\cventry{2022-2023}{Technological innovation cell}{Iberdrola Neoenergia COSERN, Brazil}{R\$359,234.61}{Proponent}{}

\cventry{2021-2023}{Information technology postgraduate apprenticeship (class of 2023)}{5th Region Federal Regional Court (TRF5), Brazil}{R\$2,816,840.00}{Collaborator}{}

\cventry{2020-2021}{Applied research and human resource education in hardware technologies for artificial intelligence}{Huawei Telecommunications in Brazil}{R\$455,375.00}{Proponent}{}

\cventry{2017-2018}{SmartMetropolis}{Multiple local and national government branches, Brazil}{R\$3,609,907.74}{Collaborator}{}

\cventry{2017-2018}{Revision and update of the Brazilian Standard Classification of Education (CINE Brasil 2018)}{UNESCO \& Brazilian Ministry of Education -- INEP, Brazil}{R\$1,000,000.00}{Proponent}{}

\cventry{2015-2016}{Combinatorial optimization: metaheuristics and exact methods (COMEX)}{Belgian Federal Science Policy Office (BELSPO), Belgium}{€500,000.00}{Collaborator}{}

\cventry{2011-2015}{Generalization of metaheuristics for optimization problems with three or more objectives}{Fonds de la Recherche Scientifique (FNRS), Belgium}{€100,000.00}{Proponent}{}

%----------------------------------------------------------------------------------------
%	PUBLICATIONS
%----------------------------------------------------------------------------------------

\section{Key$\star$ and relevant publications}\closesection
\cvline{disclaimer}{\textit{An exhaustive publication list with full author description is provided at the end of the document.}}
\subsection{Journals (5)}
\cventry{2021}{A computational study on ant colony optimization for the traveling salesman problem with dynamic demands}{Computers \& Operations Research}{h-index: 160}{}% IF: 5.159,
{This paper was the main contribution from the first Ph.D. thesis I co-supervised, and demonstrates how multi-objective and dynamic optimization intersect. The relevance of this paper is evidenced by its best paper award nomination at the EMO 2019 conference, where a preliminary version of the journal paper was first published. In addition, this paper is a concrete example of how I bridge different research topics into multi-disciplinary work.}  

\cventry{2021}{Comparing community mobility reduction between first and second COVID-19 waves}{Transport Policy}{h-index: 103}{}% IF: 6.173, 
{This paper was the main contribution of my efforts in science publication and communication to assist in the fight against the COVID-19 pandemic. Indeed, the first author of this paper is one of the undergraduate students that I helped mobilize in those initiatives. The relevance of this paper is evidenced by the number of different continents and COVID-19 waves included in the assessment. In addition, this paper is a concrete example of how I use computational intelligence in the context of socially relevant problems. }

% \cventry{2020}{\#StayHome: Monitoring and benchmarking social isolation trends in Caruaru and the Região Metropolitana do Recife during the COVID-19 pandemic}{Revista da Sociedade Brasileira de Medicina Tropical}{IF: 1.581}{}{}
\cvline{}{\hrule}
\cventry{$\star$2020}{Automatically designing state-of-the-art multi-and many-objective evolutionary algorithms}{MIT Evolutionary Computation Journal}{h-index: 84}{}% IF: 4.766,
{}
\cventry{$\star$2018}{A large-scale experimental evaluation of high-performing multi-and many-objective evolutionary algorithms}{MIT Evolutionary Computation Journal}{h-index: 84}{}% IF: 4.766, 
{}
\cventry{$\star$2016}{Automatic component-wise design of multiobjective evolutionary algorithms}{IEEE Transactions on Evolutionary Computation}{h-index: 186}{}% IF: 16.497,
{These papers comprise the contributions of my Ph.D. thesis, having been accepted for publication prior to my defense or shortly after. Their relevance is evidenced by their ongoing impact on the evolutionary computation community, one of the most important in the context of CI, and by the rigorous journals where they were published. More importantly, these papers demonstrate how I am able to plan and deliver on a research project. In detail, each paper meets an specific objective of my thesis proposal, incrementally achieving the general objective of the project.}  


% \cventry{2013}{Analyzing the impact of MOACO components: An algorithmic study on the multi-objective shortest path problem}{Expert Systems with Applications}{IF: 8.665}{}{}

% \subsection{Book chapters (1)}
% \cventry{$\star$2020}{Automatic configuration of multi-objective optimizers and multi-objective configuration}{High-Performance Simulation-Based Optimization}{SCI}{Springer}{}

\subsection{Conference papers (11)}

\cventry{2022}{High school timetabling at a federal educational institute in Brazil}{IEEE WCCI}{}{}{}
\cventry{2022}{Retail sales forecasting for a Brazilian supermarket chain: an empirical assessment}{IEEE CBI}{}{}{}
\cventry{2022}{Supermarket customer segmentation: a case study in a large Brazilian retail chain}{IEEE CBI}{}{}{}
\cventry{$\star$2018}{Time-series features for predictive policing}{IEEE ISC2}{}{}{}
\cventry{2018}{Towards a crime hotspot detection framework for patrol planning}{IEEE SmartCity}{}{}
{These papers comprise the contributions of data science M.Sc. theses I (co-)supervised in partnership with public and private institutions. The relevance of these papers is evidenced by the socially relevant scenarios they address. In detail, the first paper focuses on the Brazilian Federal Network of Vocational, Scientific and Technological Education, which provides education to over two million students, with over half of the students that declared income, gender, and ethnicity coming from low income families, being women, and self-declaring as non-white. The remainder 2022 papers use AI techniques to model different business processes in the 3rd largest retail supermarket chain in the Northeast of Brazil, and is instrumental to assess the impact of the COVID-19 pandemic in the industry. Finally, the 2018 papers address predictive policing to assist the local government in the forecasting of criminal occurrences.}
\cvline{}{\hrule}

\cventry{2021}{Evaluating anytime performance on NAS-Bench-101}{IEEE CEC}{}{}{}
\cventry{$\star$2021}{iSklearn: automated machine learning with irace}{IEEE CEC}{}{}{}
\cventry{2021}{Comparing contextual embeddings for semantic textual similarity in Portuguese}{BRACIS}{}{}%
{These papers are the contributions of M.Sc. theses I supervised in deep and automated machine learning. The relevance of these papers is evidenced by the state-of-the-art techniques that were employed. In addition, the application domains considered are among the most relevant that use unstructured data, namely computer vision, natural language processing, and time series forecasting. Importantly, these papers demonstrate that I understand the technological complexity of current state-of-the-art AI models, their potential impact on society, and therefore their need for accountability.}

\cvline{}{\hrule}

\cventry{2021}{Revisiting Pareto-optimal multi-and many-objective reference fronts for continuous optimization}{IEEE CEC}{}{}{}
\cventry{$\star$2019}{Archiver effects on the performance of state-of-the-art multi-and many-objective evolutionary algorithms}{GECCO}{}{}{}
\cventry{$\star$2017}{An empirical assessment of the properties of inverted generational distance on multi-and many-objective optimization}{EMO}{}{}%
{These papers comprise follow-up works on my Ph.D. thesis. The relevance of these papers is evidenced by their ongoing impact on the evolutionary computation community, as well as the conferences where they were published, which are among the top-tier venues in their field. More importantly, these papers are a concrete example that the work I conducted in my Ph.D. was seminal to relevant future work. In addition, they demonstrate that I understand that seeking autonomy as an independent researcher does not mean discontinuing previous research.}

% \cvline{}{\hrule}

% \cventry{2019}{The hypervolume indicator as a performance measure in dynamic optimization}{EMO}{}{}{}




% \cventry{2015}{To DE or not to DE? Multi-objective differential evolution revisited from a component-wise perspective}{EMO}{}{}{}

% \cventry{2015}{Comparing decomposition-based and automatically component-wise designed multi-objective evolutionary algorithms}{EMO}{}{}{}

% \cventry{$\star$2014}{Automatic design of evolutionary algorithms for multi-objective combinatorial optimization}{PPSN}{}{}{}

% \cventry{2014}{Deconstructing multi-objective evolutionary algorithms: An iterative analysis on the permutation flow-shop problem}{LION}{}{}{}

% \cventry{2013}{An analysis of local search for the bi-objective bidimensional knapsack problem}{EvoCOP}{}{}{}

% \cventry{2012}{Automatic generation of multi-objective ACO algorithms for the bi-objective knapsack}{ANTS}{}{}{}

% \cventry{2011}{GRACE: a generational randomized ACO for the multi-objective shortest path problem}{EMO}{}{}{}

% \cventry{2009}{FAITH: a desktop virtual reality system for fingerspelling}{SVR}{}{}{}

%----------------------------------------------------------------------------------------
%	SUPERVISION
%----------------------------------------------------------------------------------------

\section{Supervision}

% \subsection{Ph.D. theses}

\cventry{2022}{Design configuration for the MMAS algorithm applied to the travelling salesman problem with dynamic demands}{Ph.D. thesis co-supervisor}{Sabrina M. de Oliveira}{Computational mathematical modeling, Federal Center of Technological Education of Minas Gerais (CEFET-MG), Brazil}{}

% \subsection{M.Sc. theses}

\cventry{2021}{A case study on customer segmentation of a supermarket chain}{M.Sc. thesis supervisor}{Wellerson V. Oliveira}{Information technology, Federal University of Rio Grande do Norte (UFRN), Brazil}{}

\cventry{2021}{Sales forecasting for a supermarket chain in Natal, Brazil: an empirical assessment}{M.Sc. thesis supervisor}{Fernanda M. de Almeida}{Information technology, Federal University of Rio Grande do Norte (UFRN), Brazil}{}{}

\cventry{2021}{Assessing irace for automated machine and deep learning in computer vision}{M.Sc. thesis supervisor}{Carlos E. M. Vieira}{Information technology, Federal University of Rio Grande do Norte (UFRN), Brazil}{}{}

\cventry{2018}{Uma abordagem metaheurística para o problema de alocação de horário escolar no IFRN}{M.Sc. thesis supervisor}{Lucas H. A. Dantas}{Systems and computing, Federal University of Rio Grande do Norte (UFRN), Brazil}{}{}

\cventry{2018}{Predizendo hotspots criminais com aprendizado de máquina}{M.Sc. thesis co-supervisor}{Adelson D. de Araújo Júnior}{Systems and computing, Federal University of Rio Grande do Norte (UFRN), Brazil}{}{}

% \subsection{B.Sc. theses}

% \cventry{2021}{Comparando embeddings contextuais no problema de busca de similaridade semântica em português}{Supervisor}{Information technology}{Federal University of Rio Grande do Norte (UFRN)}{}{}

% \cventry{2019}{Assessing irace for automated machine learning}{Supervisor}{Computer science}{Federal University of Rio Grande do Norte (UFRN)}{}{}

% \cventry{2019}{Using artificial intelligence to aid depression detection}{Supervisor}{Computer engineering}{Federal University of Rio Grande do Norte (UFRN)}{}{}


%----------------------------------------------------------------------------------------
%	LANGUAGES SECTION
%----------------------------------------------------------------------------------------

\section{Languages}

\cvlanguage{Portuguese}{Fluent}{Mother language}
\cvlanguage{English}{Fluent}{TOEFL iBT Score 108}
\cvlanguage{Spanish}{Fluent}{European Union Reference Level C1}
\cvlanguage{French}{Intermediate}{European Union Reference Level B1}
\cvlanguage{Italian}{Intermediate}{European Union Reference Level B1}

% \cvcomputer{category}{programs}{category}{programs}
% \cvline{leftmark}{text}
% \cvdoubleitem{subtitle}{text}{subtitle}{text}
% \cvlistitem{point1}
% \cvlistdoubleitem{point1}{point2}

% \nocite{*}
% \bibliographystyle{plainyr-rev}
% \bibliography{journals,chapters}

% \bibliographyunit[\section]
% \begin{bibunit}
% \bibliography{journals}
% \putbib
% \end{bibunit}

% \begin{bibunit}
% \bibliography{chapters}
% \putbib
% \end{bibunit}

\section{Publication list}
\nociteJournals{*}
\bibliographystyleJournals{plainyr-rev}
\bibliographyJournals{bib/journals}

\nociteChapters{*}
\bibliographystyleChapters{plainyr-rev}
\bibliographyChapters{bib/chapters}

\nociteConferences{*}
\bibliographystyleConferences{plainyr-rev}
\bibliographyConferences{bib/conferences}
%-----------------------------------------------------------------
-----------------------
%	COMPUTER SKILLS SECTION
%----------------------------------------------------------------------------------------

% \section{Skills}

% \cvitem{Softwares:}{MATLAB, Excel/Office, STWAVE (modelling) PTC Creo (Limited), Python (Limited)}
% \subsection{Equipment/Hardware Skills}
% \cvitem{Wave Tank:}{Hydraulic Wavemaker, Load cell, Wave-guage operation}
% \cvitem{Flow Map:}{Flow Tank, Dye pumps (visualization), Class IV Laser operation}
% \cvitem{Acoustics:}{Oscilloscope, Function Generator, Humminbird 1158c Transducer}


\end{document}